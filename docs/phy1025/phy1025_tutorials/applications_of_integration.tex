\documentclass{article}
\usepackage{amsmath}
\usepackage{amssymb}
\usepackage{hyperref}
\usepackage{physics}

\title{Applications of Integration}
\author{James Capers \thanks{jrc232@exeter.ac.uk}}
\date{\today}

\begin{document}

\maketitle


\section{Surfaces, Areas and Volumes}

\begin{enumerate}
	\item{Find the length of the curve $y= \frac{x}{2} - \frac{x^2}{4} +\frac{1}{2} \ln(1-x)$ between $x=0$ and $x=1/2$.
		\begin{align*}
			\dv{y}{x} &= \frac{1}{2} - \frac{x}{2} - \frac{1}{2(x-1)} \\
			&= \frac{1}{2} \left( 1 - x - \frac{1}{x-1} \right) \\
			&= \frac{(1-x)^2 - 1}{2(1-x)} \\
			\left( \dv{y}{x} \right)^2 &= \frac{(1-x)^4 - 2(1-x)^2 + 1}{4(1-x)^2} \\
			l &= \int_0^{1/2} \sqrt{1 + \frac{(1-x)^4 - 2(1-x)^2 + 1}{4(1-x)^2} } dx\\
			&= \int_0^{1/2} \sqrt{ \frac{4(1-x)^2 + (1-x)^4 - 2(1-x)^2 + 1}{4(1-x)^2} } dx \\
			&= \int_0^{1/2} \frac{\sqrt{\left[ (1-x)^2 + 1 \right]^2}}{2(1-x)} dx\\
			&= \int_0^{1/2} \frac{(1-x)^2 + 1}{2(1-x)} dx \\
			&= \frac{1}{2} \int_0^{1/2} \left( 1-x + \frac{1}{1-x} \right) dx \\
			&= \frac{1}{2} \left[ x - \frac{x^2}{2} - \ln (1-x) \right]_0^{1/2} \\
			l &= \frac{3}{16} + \frac{1}{2} \ln 2
		\end{align*}			
	}
	\item{The parametric equations of a curve are $x = e^t \sin t$,$y = e^t \cos t$. If the arc of this curve, between $t = 0$ and $t = \frac{\pi}{2}$, rotates through a complete revolution about the $x$-axis, calculate the area of the surface generated.
		\begin{align*}
			A &= \int_{t_1}^{t_2} 2 \pi y \sqrt{\left( \dv{x}{t} \right)^2 + \left( \dv{y}{t} \right)^2 } dt \\
			\dv{x}{t} &= e^t (\sin t + \cos t) & \dv{y}{t} &= e^t (\cos t - \sin t) \\
			\left( \dv{x}{t} \right)^2 + \left( \dv{y}{t} \right)^2 &= 2 e^{2t} \\
			A &= \int_0^{\pi/2} 2 \pi e^t \cos t \sqrt{2} e^t dt \\
			&= 2\sqrt{2} \pi \int_0^{\pi/2} e^{2t} \cos t dt .
		\end{align*}		
		This can now be integrated by parts, with $u = e^{2t}$, $du = 2e^{2t}$, $dv = \cos t$ and $v = \sin t$.  
		\begin{align*}
			I &= \int_0^{\pi/2} e^{2t} \cos t dt \\
			&= \left[ e^{2t} \sin t \right]_0^{\pi/2} - 2 \int_0^{\pi/2} e^{2t} \sin t dt \\
			&= e^{\pi} - 2 \int_0^{\pi/2} e^{2t} \sin t dt .
		\end{align*}			
		We need to integrate by parts again, this time with $u = e^{2t}$, $du = 2 e^{2t}$, $dv = \sin t$, $v = -\cos t$.
		\begin{align*}
			I &= e^{\pi} + 2\left[ e^{2t} \cos t\right]_0^{\pi/2} - 4 \int_0^{\pi/2} e^{2t} \cos t dt \\
			&= e^{\pi} - 2 - 4 I \\
			I &= \frac{e^\pi - 2}{5} .
		\end{align*}
		We then have the final answer
		\begin{equation*}
			A = \frac{2 \sqrt{2} \pi (e^\pi - 2)}{5} .
		\end{equation*}
	}
	\item{The line $y = 2x$ and the parabola $y^2 = 16x$ intersect at $x=4$.  Find, by double integral, the area enclosed by $y = 2x$, $y^2 = 16x$, $x=1$ and the point of intersection $x=4$.
		\begin{align*}
			A &= \int_0^4 dx \int_{2x}^{4\sqrt{x}} dy \\
			&= \int_0^4 dx \left( 4\sqrt{x} - 2x \right) \\
			&= \left[ \frac{8}{3} x^{3/2} - x^2 \right]_0^4 \\
			&= \frac{11}{3} .
		\end{align*}			
	}
	\item{Determine the area bounded by the curves $x=y^2$ and $x = 2y - y^2$.\\
		The first step is to find the point of intersection.  By equating the two expressions $y^2 = 2y - y^2$ we find that the intersection points are at $y = 0$ and $y=1$.
		Now, to find the area
		\begin{align*}
			A &= \int_0^1 dy \int_{y^2}^{2y-y^2} dx \\
			&= \int_0^1 dy 2y - y^2 \\
			&= \left[ y^2 -\frac{2}{3} y^3 \right]_0^1 \\
			&= \frac{1}{3} .
		\end{align*}
	}
	\item{A rectangular block is bounded by the co-ordinate planes of reference and the planes $x = 3$, $y = 4$, $z = 2$. Its density at any point is numerically equal to the square of its distance from the origin. Find the total mass of the solid. \\
		The density at any point is $\rho = x^2 + y^2 + z^2$ so the total mass is
		\begin{align*}
			m &= \int \rho dV \\
			&= \int_0^3 dx \int_0^4 dy \int_0^2 dz  (x^2 + y^2 + z^2) \\
			&= \int_0^2 \int_0^4 dy dz \left[ \frac{x^3}{3} xy^2 + xz^2 \right]_0^3 \\
			&= \int_0^2 \int_0^4 dy dz (9 + 3y^2 + 3z^2) \\
			&= \int_0^2 dz \left[ 9y + y^3 + 3y z^2 \right]_0^4 \\
			&= \int_0^2 dz (100 + 12z^2) \\
			&= \left[ 100z + 4z^3 \right]_0^2 \\
			&= 232 .
		\end{align*}
	}
\end{enumerate}

\section{Multiple Integrals}

\begin{enumerate}
	\item{Evaluate 
		\begin{equation*}
			I = \int_0^a dx \int_0^{y_i} dy (x-y) ,
		\end{equation*}			
		where $y_i = \sqrt{a^2 - x^2}$.
		\begin{align*}
			I &= \int_0^a \left[ xy - \frac{y^2}{2} \right]_0^{y_i} \\
			&= \int_0^a x\sqrt{a^2 - x^2} - \frac{a^2 - x^2}{2} dx \\
			&= \frac{1}{2} \left[ -\frac{2}{3} (a^2 - x^2)^{3/2} - a^2 x + \frac{x^3}{3}\right]_0^a \\
			&= \frac{1}{2} \left( \frac{2a^3}{3} - a^3 + \frac{a^3}{3} \right) = 0.
		\end{align*}
	}
	\item{Evaluate 
		\begin{equation*}
			I = \int_0^a \int_0^b \int_0^c (x^2 + y^2) dx dy dz.
		\end{equation*}			
		\begin{align*}
			I &= \int_0^a \int_0^b \left[ \frac{x^3}{3} + xy^2 \right]_0^c dy dz \\
			&= \int_0^a \int_0^b \left( \frac{c^3}{3} + cy^2 \right) dy dz \\
			&= \int_0^a \left[ \frac{c^3 y}{3} + \frac{cy^3}{3}\right]_0^b dz\\
			&= \int_0^a \left( \frac{c^3 b}{3} + \frac{cb^3}{3} \right) dz \\
			&= \frac{abc}{3} (c^2 + b^2).
		\end{align*}
	}
	\item{Evaluate 
		\begin{equation*}
			I = \int_0^{\pi} \int_0^{\pi/2} \int_0^r x^2 \sin \theta dx d\theta d\phi .
		\end{equation*}		
		\begin{align*}
			I &= \int_0^\pi \int_0^{\pi/2} \left[ \frac{x^3}{3} \sin \theta \right]_0^r d\theta d\phi \\
			&= \int_0^\pi \int_0^{\pi/2} \left( \frac{r^3}{3} \sin \theta \right) d\theta d\phi \\ 
			&= \int_0^\pi \left[ -\frac{r^3}{3} \cos \theta \right]_0^{\pi/2} d\phi \\
			&= \int_0^\pi \frac{r^3}{3} d\phi \\
			&= \frac{\pi r^3}{3} .
		\end{align*}			
	}
\end{enumerate}

\end{document}