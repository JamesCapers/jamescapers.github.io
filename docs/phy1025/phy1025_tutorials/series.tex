\documentclass{article}
\usepackage{amsmath}
\usepackage{amssymb}
\usepackage{hyperref}
\usepackage{physics}

\title{Series}
\author{James Capers \thanks{jrc232@exeter.ac.uk}}
\date{\today}

\begin{document}

\maketitle

\begin{abstract}
Most of the questions this week are taken from chapter 5 of ``Mathematical Methods for Physicists'' by Arfken and Weber.
\end{abstract}

In the questions below, we'll use the following definitions.
The Taylor series is an expansion of a function about a point, $a$ and is defined as
\begin{equation}
	f(x) = f(a) + f'(a) (x-a) + \frac{f''(a)}{2!} (x-a)^2 + \frac{f'''(a)}{3!} (x-a)^3 + \frac{f''''(a)}{4!} (x-a)^4 + \ldots
\end{equation}
The special case of $a = 0$ is called a Maclaurin series.

Binomial series is defined as
\begin{equation}
	(1+x)^m = \sum_{n=0}^\infty \frac{m!}{n!(m-n)!} x^n .
\end{equation}
As we will see, this can be derived by applying the Maclaurin series to $(1+x)^m$.

\begin{enumerate}
	\item{Find the Maclaurin series of 
		\begin{equation*}
			f(x) = e^x .
		\end{equation*}
		The derivative of $e^x$ is $e^x$ and this evaluated at $x=0$ is $1$, making the series easy to compute. \\
		\begin{align*}
			e^x &= e^0 + x e^0 + \frac{x^2}{2!} e^0 + \frac{x^3}{3!} e^0 + \ldots \\
			&= 1 + x + \frac{x^2}{2} + \frac{x^3}{6} + \ldots 
		\end{align*}
	}
	\item{Find the Maclaurin series of 
		\begin{equation*}
			f(x) = \ln (1+x) ,
		\end{equation*}
		and show that for $x=1$ this becomes the harmonic series $\sum_{n=1}^\infty (-1)^{n-1} n^{-1}$. \\
		We begin by finding the series expansion.
		\begin{align*}
			f(0) &= 0 \\
			f'(x) &= \frac{1}{1+x} & f'(0) &= 1 \\
			f''(x) &= \frac{11}{(1+x)^2} & f''(0) &= -1 \\
			f'''(x) &= \frac{2}{(1+x)^3} & f'''(0) &= 2 \\
			f''''(x) &= \frac{-6}{(1+x)^4} & f''''(0) &= -6 \\
			f'''''(x) &= \frac{25}{(1+x)^5} & f'''''(0) &= 24 \\
		\end{align*}
		so that 
		\begin{equation*}
			\ln (1+x) = x - \frac{x^2}{2} + \frac{x^3}{3} - \frac{x^4}{4} + \frac{x^5}{5} + \ldots
		\end{equation*}
		Then, for $x=1$, this becomes
		\begin{equation*}
			\ln (2) = 1 - \frac{1}{2} + \frac{1}{3} - \frac{1}{4} + \frac{1}{5} + \ldots
		\end{equation*}
		which can be written as $\sum_{n=1}^\infty (-1)^{n-1} n^{-1}$.
	}
	\item{The total relativistic energy of a particle of mass $m$ and velocity $v$ is 
		\begin{equation*}
			E = mc^2 \left( 1 - \frac{v^2}{c^2} \right)^{-1/2}.
		\end{equation*}
	Compare this with the classical kinetic energy $mv^2 / 2$. \\
	To make the comparison, let's say that $x = v^2/c^2$ and expand $(1-x)^{-1/2}$ for $x = 0$.  This corresponds to the limit where $v \ll c$, where we expect the effects of relativity to be small.  Making the expansion 
		\begin{align*}
			f(0) &= 1 \\
			f'(x) &= \frac{1}{2} (1-x)^{-3/2} & f'(0) &= \frac{1}{2} \\
			f''(x) &= \frac{3}{4} (1-x)^{-5/2} & f''(0) &= \frac{3}{4} \\
			f'''(x) &= \frac{15}{8} (1-x)^{-7/2} & f'''(0) &= \frac{15}{8}  
		\end{align*}
		so that the energy expansion, in terms of $v^2/c^2$ is 
		\begin{align*}
			E &= mc^2 \left[ 1 + \frac{1}{2} \left( \frac{v^2}{c^2} \right) + \frac{3}{8} \left( \frac{v^2}{c^2} \right)^2 + \frac{15}{8} \frac{1}{3!} \left( \frac{v^2}{c^2} \right)^3 + \ldots \right] \\
			&= mc^2 + \frac{1}{2} m v^2 + \ldots 
		\end{align*}
		The first term is the mass--energy and the second term is the usual classical kinetic energy.  All of the higher terms represent relativistic corrections to the energy as $v \rightarrow c$.
	}
	\item{By applying the Maclaurin series to $(1+x)^m$, derive the Binomial series. \\
	Evaluating the series, we have 
		\begin{align*}
			f(0) &= 1 \\
			f'(x) &= m (1+x)^{m-1} & f'(0) &= m \\
			f''(x) &= m (m-1) (1+x)^{m-2} & f''(0) &= m (m-1) \\
			f'''(x) &= m (m-1) (m-2) (1+x)^{m-3} & f'''(0) &= m (m-1) (m-2) .
		\end{align*}
		Putting this together, we have 
		\begin{equation*}
			(1+x)^m = 1 + mx + m (m-1) \frac{x^2}{2} + m(m-1)(m-2) \frac{x^3}{3!} + \ldots
		\end{equation*}
		Comparing this with the binomial series, which is usually written as
		\begin{align*}
			(1+x)^m &= \sum_{k=0}^\infty 
			\begin{pmatrix}
				m \\ k
			\end{pmatrix}
			x^k, & \begin{pmatrix}
				m \\ k
			\end{pmatrix} 
			&= \frac{m (m-1) (m-2) \ldots (m -k +1)}{k!}
		\end{align*}
		we can see that the two are identical.
	}
	\item{Derive the geometric series by expanding
		\begin{equation*}
			f(x) = \frac{1}{1-x}.
		\end{equation*}
		around $x=0$. \\
		Evaluating the series, we have 
		\begin{align*}
			f(0) &= 1 \\
			f'(0) &= \frac{1}{(1-x)^2} & f'(0) &= 1 \\
			f''(0) &= \frac{2}{(1-x)^3} & f''(0) &= 2 \\
			f'''(0) &= \frac{6}{(1-x)^4} & f'''(0) &= 6 \\
			f''''(0) &= \frac{24}{(1-x)^5} & f''''(0) &= 24
		\end{align*}
		so that 
		\begin{equation*}
			(1-x)^{-1} = 1 + x + x^2 + x^3 + x^4 + \ldots 
		\end{equation*}
		This is the geometric series and clearly only converges when $|x| < 1$.  As long as this condition is met, each term is smaller than the one before it and the sum converges.  
	}
	\item{}
	\begin{enumerate}
		\item{Given that 
			\begin{equation*}
				\ln (1+x) = x - \frac{x^2}{2} + \frac{x^3}{3} - \frac{x^4}{4} + \ldots
			\end{equation*}
			show that 
			\begin{equation*}
				\ln \left( \frac{1+x}{1-x} \right) = 2\left( x + \frac{x^3}{3} + \frac{x^5}{5} + \ldots \right) .
			\end{equation*}
			We begin by noting that $\ln \left( \frac{1+x}{1-x} \right) = \ln (1+x) - \ln (1-x)$.  Now, since we know the expansion of $\ln (1+x)$, we can find the expansion for $\ln (1-x)$ without any differentiation, by just replacing $x \rightarrow -x$ in the expansion.  This gives us 
			\begin{equation*}
				\ln (1-x) = -x - \frac{x^2}{2} - \frac{x^3}{3} - \frac{x^4}{4} + \ldots
			\end{equation*}
			so that 
			\begin{align*}
				\ln \left( \frac{1+x}{1-x} \right) &= \left( x - \frac{x^2}{2} + \frac{x^3}{3} - \frac{x^4}{4} \right) - \left( -x - \frac{x^2}{2} - \frac{x^3}{3} - \frac{x^4}{4} \right) \\
				&= 2\left( x + \frac{x^3}{3} + \frac{x^5}{5} + \ldots \right)  .
			\end{align*}
		}
		\item{Expand $f(x) = \arctan x$ around $x=0$. \\
		Evaluating the series, we have 
			\begin{align*}
				f(0) &= 1 \\
				f'(x) &= (1+x^2)^{-1} & f'(0) &= 1 \\
				f''(x) &= -2x(1+x^2)^{-2} & f''(0) &= 0 \\
				f'''(x) &= 8x^2 (1+x^2)^{-3} - 2(1+x^2)^{-2} & f'''(0) &= -2 \\
				f''''(x) &= -48x^3 (1+x^2)^{-4} + 24x (1+x^2)^{-2} & f''''(0) &= 0 \\
				f'''''(x) &= 384x^4 (1+x^2)^{-5} -288x^2 (1+x^2)^{-4} + 24 (1+x^2)^{-3} & f'''''(0) &= 24 .
			\end{align*}
			Giving 
			\begin{equation*}
				\arctan (x) = 1 - \frac{x^3}{3} + \frac{x^5}{5} + \ldots
			\end{equation*}
		}
		\item{Expand using the binomial theorem 
			\begin{equation*}
				f(t) = \frac{1}{1+t^2}.
			\end{equation*}
			Expanding using the binomial theorem we wrote down in question 4, replacing $x$ with $t^2$, we get
			\begin{equation*}
				(1+t^2)^{-1} = 1 - t^2 + t^4 - t^6 + t^8 + \ldots
			\end{equation*}
		}
		\item{Using this expansion, integrate term by term to show that 
			\begin{equation*}
				\arctan x = \int_0^x \frac{dt}{1+t^2} = \sum_{n=0}^\infty (-1)^n \frac{x^{2n + 1}}{2n + 1} .
			\end{equation*}
			Integrating term by term, we get
			\begin{align*}
				\int_0^x \frac{dt}{1+t^2} &= \int_0^x dt - \int_0^x dt \ t^2 + \int_0^x dt \ t^4 + \ldots \\
				&= \left[ t \right]_0^x - \left[ \frac{t^3}{3} \right]_0^x + \left[ \frac{t^5}{5} \right]_0^x + \ldots \\
				&= x - \frac{x^3}{3} + \frac{x^5}{5} + \ldots \\
				&= \arctan (x) .
			\end{align*}
		}
		\item{By comparing the series, show that 
			\begin{equation*}
				\arctan x = \frac{i}{2} \ln \left(\frac{1-ix}{1+ix}\right).
			\end{equation*}
			We know from part (a) that 
			\begin{equation*}
				\ln \left( \frac{1+x}{1-x} \right) = 2 \left( x + \frac{x^3}{3} + \frac{x^5}{5} + \ldots \right) 
			\end{equation*}
			so using log laws we have
			\begin{equation*}
				\ln \left( \frac{1-x}{1+x} \right) = - 2 \left( x + \frac{x^3}{3} + \frac{x^5}{5} + \ldots \right) .
			\end{equation*}
			Now, we know that 
			\begin{align*}
				\frac{i}{2} \ln \left( \frac{1-ix}{1+ix} \right) &= - 2\frac{i}{2} \left( ix + \frac{i^3 x^3}{3} + \frac{i^5 x^5}{5} + \ldots \right) \\
				&= \frac{1}{i} \left( ix -i \frac{x^3}{3} + i \frac{x^5}{5} + \ldots \right) \\
				&= x - \frac{x^3}{3} + \frac{x^5}{5} + \ldots \\
				&= \arctan (x) .
			\end{align*}
			We made use of the fact that $-i = 1/i$.
		}
	\end{enumerate}
\end{enumerate}

\end{document}