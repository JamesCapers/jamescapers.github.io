\documentclass{article}
\usepackage{amsmath}
\usepackage{amssymb}
\usepackage{hyperref}

\title{Integration}
\author{James Capers \thanks{jrc232@exeter.ac.uk}}
\date{\today}

\begin{document}

\maketitle

There are many ways to evaluate integrals.
Some methods you may be familiar with are:
\begin{enumerate}
	\item{Integration by inspection.  \\ 
	This sort of technique is normally used for very basic integrals like $\int x^2 dx$, $\int \sin x dx$ or $\int 1/x dx$.}
	\item{Integration by substitution.  \\
	This is when an integral that looks tricky can be brought to a form that we can integrate by inspection if we make an appropriate variable change.  Finding the correct variable change can be tricky as there is no general rule.}
	\item{Integration by parts.  \\
	When the integral can be written as a product of something that can be differentiated easily with something that can be easily integrated, integration by parts $\int u dv = uv - \int v du$ can be used to evaluate the integral. }
\end{enumerate}
In this tutorial we'll consider some examples of each.

\section{Integration by Inspection}

\subsection{Elementary Integrals}

Some integrals we can just write down the answer to.
These results can be derived from first principles, but often it's not worth doing this every time.
Examples of integrals you should just be able to write down the answer to are:
\begin{itemize}
	\item{$\int x^n dx = \frac{ x^{n+1} }{n+1} + C$.}
	\item{$\int \frac{1}{x} dx = \ln x + C$.}
	\item{$\int \cos x dx = -\sin x + C$.}
	\item{$\int \frac{1}{1+x^2} dx = \tan^{-1} x + C$.}
\end{itemize}

\subsection{Partial Fractions}

Other integrals can be evaluated by inspection after we perform some manipulations, for example using partial fractions to split the integrand into bits.
Let's do an example.

\subsubsection*{Example 1}

Consider the integral
\begin{equation}
	I = \int dx \frac{2x + 1}{x^2 - 8x + 15} .
\end{equation}
We'll start by factorising the denominator as $x^2 - 8x + 15 = (x-3)(x-5)$.
Partial fractions can now be used to split up the integrand
\begin{align*}
	\frac{2x-1}{(x-3)(x-5)} &= \frac{A}{x-3} + \frac{B}{x-5} \\
	2x-1 &= A (x-5) + B(x-3) \\
	2x -1 &= (A+B) x - (5A + 3B) \\
	\Rightarrow A+B = 2,& 5A + 3B = 1 \\
	A &= -\frac{5}{2}, \\
	B &= \frac{9}{2} .
\end{align*}
So, we can now write
\begin{align*}
	\int dx \frac{2x + 1}{x^2 - 8x + 15} &= \int dx \frac{-5}{2(x-3)} + \int dx \frac{9}{2(x-5)} \\
	&= -\frac{5}{2} \ln (x-3) + \frac{9}{2} \ln (x-5) + C.
\end{align*}

\subsubsection*{Example 2}

For another example, we'll evaluate 
\begin{equation}
	\int dx \frac{2x^2 + x + 1}{(x-1)(x^2 +1)} .
\end{equation}
As before, we begin by splitting the integrand into partial fractions
\begin{align*}
	2x^2 + x + 1 &= A(x-1) + B(x^2+1) \\
	2x^2 + x + 1 &= B x^2 + A x + (B-A) \\
	\Rightarrow B = 2, & A = 1 .
\end{align*}
Now, we can integrate these partial fractions by inspection, giving
\begin{align*}
	\int dx \frac{2x^2 + x + 1}{(x-1)(x^2 +1)} &= \int dx \frac{1}{x^2 + 1} + \int dx \frac{2}{x-1} \\
	&= \tan^{-1} (x) + 2 \ln (x-1) + C.
\end{align*}

\section{Integration by Substitution}

Sometimes changing the variable of integration via a suitable substitution can make the integral \emph{much} easier to evaluate.
In this section, we'll go over a few examples of this.

\subsubsection*{Example 3}

\begin{equation}
	\int dx e^{\cos x} \sin x dx.
\end{equation}
Let's try the substitution $u = \cos x$, so $du = -\sin x dx$
\begin{align*}
	\int dx e^{\cos x} \sin x dx &= \int e^u \sin x \frac{-1}{\sin x} du \\
	&= -\int du e^u \\
	&= -e^u + C \\
	&= -e^{\cos x} + C.
\end{align*}

\subsubsection*{Example 4}

As another example, let's consider
\begin{equation}
	\int dx \frac{5}{3x - 2}.
\end{equation}
We'll try the substitution $u = 3x -2$, $du = 3 dx$
\begin{align*}
	\int dx \frac{5}{3x - 2} &= \frac{5}{u} \frac{du}{3} \\
	&= \frac{5}{3} \int \frac{du}{u} \\
	&= \frac{5}{3} \ln u + C \\
	&= \frac{5}{3} \ln (3x - 2) + C.
\end{align*}

\subsubsection*{Example 5}

As a final example of integration by substitution we'll consider 
\begin{equation}
	\int \frac{dx}{1 + 2 \sin^2 x}.
\end{equation}
For this, we'll need the following trigonometric identities \footnote{These can be looked up, for example on wikipedia \url{https://en.wikipedia.org/wiki/List_of_trigonometric_identities}}
\begin{itemize}
	\item{$\sec^2 x = 1 + \tan^2 x$.}
	\item{$\sin x = \frac{\tan x}{\sqrt{1 + \tan^2 x}}$.}
	\item{$\frac{d}{dx} \tan x = \sec^2 x$.}
\end{itemize}
To evaluate the integral, we'll use the substitution $u = \tan x$ so $du = \sec^2 x dx = (1 + \tan^2 x) dx = (1 + u^2) dx$.
We'll also re-write $\sin^2 x$ in terms of this substitution as
\begin{equation*}
	\sin x = \frac{\tan x}{\sqrt{1 + \tan^2 x}} = \frac{u}{\sqrt{1 + u^2}}.
\end{equation*}
Putting this together to make the substitution
\begin{align*}
	\int \frac{dx}{1 + 2 \sin^2 x} &= \int \frac{du}{1 + u^2} \frac{1}{1 + \frac{2u^2}{1+u^2}} \\
	&= \int du \frac{1}{1 + 3 u^2} .
\end{align*}
If we make a final substitution $y = \sqrt{3} u$, $dy = \sqrt{3} du$, then
\begin{align*}
	\int \frac{dx}{1 + 2 \sin^2 x} &= \int du \frac{1}{1 + 3 u^2} \\
	&= \frac{1}{\sqrt{3}} \int du \frac{1}{1 + y^2} \\
	&= \frac{1}{\sqrt{3}} \tan^{-1} (y) + C.
\end{align*}
All that remains is to undo the substitutions to write the answer in terms of $x$
\begin{equation*}
	\int \frac{dx}{1 + 2 \sin^2 x} = \frac{1}{\sqrt{3}} \tan^{-1} (\sqrt{3} \tan x) + C.
\end{equation*}

\section{Integration by Parts}

The expression we'll be using to integrate by parts is 
\begin{equation*}
	\int u dv = uv - \int v du .
\end{equation*}

\subsubsection*{Example 6}

We'll begin with the definite integral 
\begin{equation}
	\int_0^1 x e^{ax} dx .
\end{equation}
We'll set $u = x$, $dv = e^{ax}$ so $du = 1$, $v = \frac{1}{a} e^{ax}$.
Plugging this into the integration by parts expression 
\begin{align*}
	\int_0^1 x e^{ax} dx &= \left. \frac{x}{a} e^{ax} \right|_0^1 - \frac{1}{a} \int_0^1  e^{ax} dx \\
	&= \frac{1}{a} e^a - \frac{1}{a^2} \left[ e^{ax} \right]_0^1 \\
	&= \frac{1}{a} e^a - \frac{1}{a^2} \left[ e^{a} -1 \right] \\
	&= \frac{e^a}{a} \left( 1 - \frac{1}{a} \right) + \frac{1}{a^2} .
\end{align*}

\subsubsection*{Example 7}

Now, let's evaluate
\begin{equation}
	\int dx e^{-3x} \cos 2x .
\end{equation}
We set $u = e^{-3x}$ so $du = -3 e^{-3x}$ and $dv = \cos 2x$ so $v = \frac{1}{2} \sin 2x$.
Plugging these into the expression for integration by parts
\begin{equation*}
	I = \frac{1}{2} e^{-3x} \sin 2x + \frac{3}{2} \int dx \sin 2x e^{-3x} .
\end{equation*}
We have to integrate this by parts again, using $u = e^{-3x}$ and $dv = \sin 2x$ so that $du = -3 e^{-3x}$ and $v = -\frac{1}{2} \cos 2x$.
Plugging in again, we get
\begin{align*}
	I &= \frac{1}{2} e^{-3x} \sin 2x - \frac{3}{4} \cos 2x e^{-3x} - \frac{9}{4} \int dx \cos 2x e^{-3x} \\
	&= \frac{1}{2} e^{-3x} \sin 2x - \frac{3}{4} \cos 2x e^{-3x} - \frac{9}{4} I \\
	I + \frac{9}{4} I &= \frac{1}{2} e^{-3x} \left[ \sin 2x - \frac{3}{2} \cos 2x \right] \\
	I &= \frac{2}{13} e^{-3x} \left[ \sin 2x - \frac{3}{2} \cos 2x \right]
\end{align*}

\subsubsection*{Example 8}

Finally, we derive a recursion formula for 
\begin{equation}
	I_n = \int \sin^n (x) dx .
\end{equation}
We begin by re--writing the integrand as $\sin x \sin^{n-1} x$ and integrating by parts with $u = \sin^{n-1} x$ and $dv = \sin x$.
This gives $v = -\cos x$ and $du = (n-1) \cos x \sin^{n-2} x$.
Plugging these into the integration by parts expression
\begin{align*}
	I_n &= -\sin^{n-1} x \cos x + (n-1) \int dx \cos^2 x \sin^{n-2} x \\
	&= -\sin^{n-1} x \cos x + (n-1) \int dx (1- \sin^2 x) x \sin^{n-2} x \\
	&= -\sin^{n-1} x \cos x + (n-1) \int dx \sin^{n-2} x - (n-1) \int dx \sin^n x \\
	I_n &= -\sin^{n-1} x \cos x + (n-1) I_{n-2} - (n-1) I_n \\
	n I_n &= -\sin^{n-1} x \cos x + (n-1) I_{n-2} .
\end{align*}

\end{document}