\documentclass{article}
\usepackage{amsmath}
\usepackage{amssymb}
\usepackage{hyperref}
\usepackage{physics}

\title{Complex Numbers}
\author{James Capers \thanks{jrc232@exeter.ac.uk}}
\date{\today}

\begin{document}

\maketitle

\begin{abstract}
Most of the questions this week are taken from chapter 2 of ``Mathematical Methods in the Physical Sciences'' by M. L. Boas (John Wiley and Sons, 1980).
The aim is to get students familiar with the basics of complex numbers and get used to manipulating them.
\end{abstract}


\begin{enumerate}
	\item{Express in polar form $z=-1-i$ and $z = (1+i)^2$. \\
		For a complex number $z = a + i b$, the magnitude is $r = \sqrt{a^2 + b^2}$ and the angle is $\theta = \arctan(y/x)$.
		These follow the usual definitions of 2D polar coordinates.	\\
		For $z=-1-i$, we have $r = \sqrt{1^2 + 1^2} = \sqrt{2}$ and $\theta = \arctan (1) = -3\pi/4$ so $z=-1-i = \sqrt{2}e^{-i3\pi/4}$.
		Now for $z=(1+i)^2$, we could expand out the brackets to get something like $z=a+ib$ or we can deal with the complex number inside the root then apply the square root - let's do the latter. \\
		If $z = 1+i$, $r = \sqrt{2}$ and $\theta = \pi/4$.  Squaring this result, we have $z=(1+i)^2 = 2e^{i\pi/2}$.
	}
	\item{Find \emph{all} of the solutions to $4x^4 - 5x^2 -6 = 0$ and plot them in the complex plane. \\
		Let's say $y = x^2$.  Now, we have $4y^2 - 5y - 6 = 0$.  We can solve this with the quadratic formula 
		\begin{align*}
			y &= \frac{-b \pm \sqrt{b^2 - 4 ac}}{2a} \\
			&= \frac{5 \pm \sqrt{25-4(4)(-6)}}{8} \\
			&= \frac{5 \pm \sqrt{25+96}}{8} \\
			&= \frac{5 \pm \sqrt{121}}{8} \\
			&= \frac{5 \pm 11}{8}
		\end{align*}			
		Taking the positive sign, we have $y = 2$ so $x = \pm \sqrt{2}$. \\
		Now, taking the minus sign, we have $y = -3/4$ so $x = \pm i \sqrt{3}{4}$. \\
		This problem is from the homework of week 1, where you could have ignored the complex roots!
	}
	\item{Find the solution to $z^2 - 2z + 2 = 0$. \\
		Plugging this into the quadratic formula, we get 
		\begin{align*}
			z &= \frac{2 \pm \sqrt{4 - 4(1)(2)}}{2} \\
			&= 1 \pm \sqrt{-1} \\
			&= 1 \pm i.
		\end{align*}			
	}
	\item{Write $1+i\sqrt{3}$ in polar form.
		\begin{equation*}
			1 + i \sqrt{3} = 2\left( \cos \frac{\pi}{3} + i \sin \frac{\pi}{3} \right) = 2e^{i\pi/3}.
		\end{equation*}	
	}	
	\item{Write 
		\begin{equation*}
			\frac{1}{2 (\cos 20^\circ + i \sin 20^\circ)}
		\end{equation*}		
		in the form $z = x+iy$. \footnote{Hint: $20^\circ = \pi/9$.}	
		\begin{align*}
			\frac{1}{2 (\cos 20^\circ + i \sin 20^\circ)} &= \frac{1}{2 (\cos \pi/9 + i \sin \pi/9)} \\
			&= \frac{1}{e^{i\pi/9}} \\
			&= 0.5 e^{-i\pi/9} \\
			&= 0.5(\cos \pi/9 - i \sin \pi/9) \\
			&= 0.47 - i 0.17 .
		\end{align*}
	}
	\item{Find 
		\begin{equation*}
			\left| \frac{\sqrt{5} + 3i}{1-i} \right| .
		\end{equation*}			
		\begin{align*}
			\left| \frac{\sqrt{5} + 3i}{1-i} \right| &= \frac{|\sqrt{5} + 3i|}{|1-i|} \\
			&= \frac{\sqrt{14}}{\sqrt{2}} = \sqrt{7} .
		\end{align*}
	}
	\item{Find all of the cube roots of 8, $\sqrt[3]{8}$. \\
		In general a complex root can be written as $z^{1/3} = r^{1/3} e^{i\theta /3}$.  We can then notice that if we write $z=8$, in polar form we get $r=8$ and $\theta$ = $0, 2\pi, 4\pi, 6\pi \ldots$ . \\
		Therefore, the cube root has $r=2$ and $\theta = 0, (2/3)\pi, (4/3)\pi$ which in degrees is $\theta = 0^\circ, 120^\circ, 240^\circ \ldots$. \\
		Therefore the cube roots of 8 are
		\begin{equation*}
			\sqrt[3]{8} = 2, -1+i\sqrt{3}, -1-i\sqrt{3} .
		\end{equation*}			
	}
	\item{Find and plot all of the values of $\sqrt[4]{-64}$. \\
		In polar coordinates $-64$ is $r = 64$ and $\theta = \pi + 2k\pi$, where $k$ is an integer.  
		Remembering that $z^{1/4} = r^{1/4} e^{i\theta/4}$ we write $\sqrt[4]{-64}$ in polar form, with $r = \sqrt[4]{64} = 2\sqrt{2}$ and $\theta = \pi/4, (\pi + 2\pi)/4, (\pi + 4\pi)/4$ and so on.  
		Simplifying, we have $\theta = \pi/4, 3\pi/4, 5\pi/4, 7\pi/4$. \\
		In rectangular form, these roots are $\pm 2 \pm 2i$.
	}	
	\item{Find and plot all of the roots of $\sqrt[6]{-8i}$. \\
		Writing $-8i$ in polar form we have $r = 8$ and $\theta = 3\pi/2 + 2\pi k$.
		The roots therefore have $r = \sqrt{2}$ and $\theta = \pi/4 + k \pi/3$.
		This gives the roots as 
		\begin{equation*}
			\pm \left\{ 1+i, \frac{\sqrt{3} + 1}{2}, - \frac{\sqrt{3} - 1}{2}, \frac{\sqrt{3} - 1}{2}, - \frac{\sqrt{3} + 1}{2} \right\} .
		\end{equation*}
	}
\end{enumerate}


\end{document}