\documentclass{article}
\usepackage{amsmath}
\usepackage{amssymb}
\usepackage{hyperref}
\usepackage{physics}

\title{Physical Applications}
\author{James Capers \thanks{jrc232@exeter.ac.uk}}
\date{\today}

\begin{document}

\maketitle

\begin{abstract}
Most of the questions this week are taken from chapter 3 of ``Mathematical Methods in the Physical Sciences'', by M. L. Boas.
\end{abstract}

\begin{enumerate}
	\item{Given the curve $y = x^2$ from $x=0$ to $x=1$, find}
	\begin{enumerate}
		\item{the area under the curve (that is, the area bounded by the curve, the $x$ axis,
		and the line $x = 1$;\\ 
		We must evaluate the integral 
			\begin{equation*}
				A = \int_0^1 dx \int_0^{x^2} dy = \int_0^1 x^2 dx = \frac{1}{3} .
			\end{equation*}		
		}
		\item{the mass of a plane sheet of material cut in the shape of this area if its density (mass per unit area) is $xy$; \\
			To find the mass, we integrate the density over the area.
			\begin{align*}
				M &= \int_0^1 dx \int_0^{x^2} dy xy \\
				&= \int_0^1 x dx \int_0^{x^2} y dy \\
				&= \int_0^1 x dx  \left[ \frac{y^2}{2} \right]_0^{x^2} \\
				&= \int_0^1 dx \ \frac{x^5}{2} \\
				&= \frac{1}{12}.
			\end{align*}
		}
		\item{the arc length of the curve; \\
			The element of length is $dl = \sqrt{dx^2 + dy^2}$, which can be re--written as $dl = \sqrt{1 + (dy/dx)^2} dx$.  To find the total length between $x=0$ and $x=1$ we need to integrate.
			\begin{align*}
				L = \int_0^1 \sqrt{1 + 4x^2} dx .
			\end{align*}
			This is a slightly tricky integral, but we can evaluate it with the substitution 
			\begin{align*}
				x &= \frac{\sinh u}{2} & dx &= \frac{\cosh u}{2} du . 
			\end{align*}
			Along the way, we'll need the double angle formulae for the hyperbolic functions, so we'll write these down now 
			\begin{align*}
				\cosh (2x) &= 2 \cosh^2 x - 1 & \sinh (2x) = 2 \sinh x \cosh x . 
			\end{align*}
			Putting in the substitution and using the fact that $\cosh^2 x - \sinh^2 x = 1$ and using the double angle formula, we have 
			\begin{align*}
				L &= \int \frac{1}{2} \cosh u du \sqrt{1 + \sinh^2 u} \\
				&= \frac{1}{2} \int du \ \cosh^2 u \\
				&= \frac{1}{4} \int du (1 + \cosh 2u) .
			\end{align*}
			This can be integrated without too much trouble, to give 
			\begin{align*}
				I = \frac{1}{4} \left( u + \frac{1}{2} \sinh 2u \right).
			\end{align*}
			Since we didn't bother to change the limits and work out the definite integral in terms of only $u$, let's undo the substitution.  The first term is easy to undo, but for the second term, we re--write $\sinh 2u = 2 \sinh u \cosh u$ and then note that $\cosh u = \sqrt{1 + \sinh^2 u}$.  This gives us 
			\begin{equation*}
				I = \frac{1}{4} \sinh^{-1} (2x) + \frac{1}{2} x \sqrt{1 + 4x^2} .
			\end{equation*}
			To find $L$, we must put in the limits.  At the lower limit, everything is zero, so putting in the upper limit $x=1$ gives us 
			\begin{equation*}
				L = \frac{1}{4} \sinh^{-1} (2) + \frac{\sqrt{5}}{2} \approx 1.48.
			\end{equation*}
		}
		\item{the center of mass; \\
		To find the centre of mass, we must evaluate the integrals 
			\begin{align*}
				\bar{x} &= \frac{1}{M} \int dA \ x \ \rho & \bar{y} &= \frac{1}{M} \int dA \ y \ \rho .
			\end{align*}
			Starting with $\bar{x}$, noting that $1/M = 12$ (from part b), we have 
			\begin{align*}
				\bar{x} &= \frac{1}{M} \int_0^1 dx \int_0^{x^2} dy x^2 y \\
				&= 12 \int_0^1 dx \left[ \frac{x^2 y^2}{2} \right]_0^{x^2} \\
				&= 12 \int_0^1 dx \frac{x^6}{2} \\
				&= \frac{12}{14} \left[ x^7 \right]_0^1 \\
				\bar{x} &= \frac{6}{7}. 
			\end{align*}
			And now the $\bar{y}$ integral
			\begin{align*}
				\bar{y} &= \frac{1}{M} \int_0^1 dx \int_0^{x^2} dy x y^2 \\
				&= 12 \int_0^1 dx \left[ \frac{x y^3}{3} \right]_0^{x^2} \\
				&= 12 \int_0^1 dx \frac{x^7}{3} \\
				&= 4 \left[ \frac{x^8}{8} \right]_0^1 \\
				\bar{y} &= \frac{1}{2} .
			\end{align*}
		}
		\item{the moments of inertia about the $x$, $y$, and $z$ axes of the lamina. \\
			To find the moments of inertia, we need to integral $\ell^2 dM$ over the object, were $\ell$ is the distance from the axis we are evaluating the moment around.  For example, the distance from the $x$ axis is $y$.  Starting with $I_x$, we have 
			\begin{align*}
				I_x &= \int_0^1 dx \int_0^{x^2} dy (xy) (y^2) \\
				&= \int_0^1 dx \left[ \frac{xy^4}{4} \right]_0^{x^2} \\
				&= \int_0^1 dx \frac{x^9}{4} \\
				&= \frac{1}{40} .
			\end{align*}
			Then, for the moment around the $y$ axis
			\begin{align*}
				I_y &= \int_0^1 dx \int_0^{x^2} dy (xy) (x^2) \\
				&= \int_0^1 dx \left[ \frac{x^3y^2}{2} \right]_0^{x^2} \\
				&= \int_0^1 dx \frac{x^7}{2} \\
				&= \frac{1}{16} .
			\end{align*}
			For the moment about the $z$ axis, the integral to evaluate is 
			\begin{align*}
				I_z &= \int_0^1 dx \int_0^{x^2} dy (xy) (x^2 + y^2) ,
			\end{align*}
			however we can note that this is just $I_x + I_y = I_z$: this is an example of the parallel axis theorem and means we can just write down $I_z = 1/16 + 1/40 = 7/80$.
		}
	\end{enumerate}
	\item{Given a semi--circular sheet ($x \geq 0$) of material of radius $a$ and constant density $\rho$, find}
	\begin{enumerate}
		\item{the centroid of the semicircular area; \\
			By symmetry, we can see that $\bar{y} = 0$ (if you're not convinced, do the integral!).  We begin by finding the mass
			\begin{align*}
				M &= \int \rho dA \\
				&= \rho \int_0^a r dr \int_{-\pi/2}^{\pi/2} d\theta \\
				&= \frac{\rho \pi a^2}{2} .
			\end{align*}					
			Notice that we could have guessed this.  A full circle would have mass $\pi a^2 \rho$ so a semi--circle has half of this.  Now, to find the $x$ coordinate of the centre of mass, we must evaluate the integral 
			\begin{align*}
				\bar{x} &= \frac{1}{M} \int x \rho dA \\
				&= \frac{2}{\rho \pi a^2} \rho \int_0^a r^2 dr \int_{-\pi/2}^{\pi/2} \cos \theta d\theta \\
				&= \frac{4a}{3\pi}.
			\end{align*}
		}
		\item{the moment of inertia of the sheet of material about the diameter forming the straight side of the semicircle. \\
		This is the moment of inertia around the $y$ axis, so the distance to this axis is the $x$ coordinate.  Written in polar coordinates this is $r \sin \theta$.  To find the moment of inertia we evaluate
			\begin{align*}
				I_y &= \int \rho x^2 dA \\
				&= \rho \int_0^a r^3 dr \int_{-\pi/2}^{\pi/2} \sin^2 \theta d\theta \\
				&= \rho \frac{a^4}{4} \int_{-\pi/2}^{\pi/2} (1 - \cos 2 \theta) d\theta \\
				&= \rho \frac{a^4 \pi}{8}.
			\end{align*}
			Note that the $\cos 2 \theta$ part of the integral does not need to be evaluated as it is an even function over symmetric limits, so the result will be zero.  
		}
	\end{enumerate}
	\item{Find the $z$ coordinate of the centroid of a solid cone of height $h$ equal to the radius of the base $r$ and uniform density $\rho$.  \\Also find the moment of inertia of the solid about its axis. \\ \\
		We begin by finding the mass, working in cylindrical coordinates and taking care that the $r$ integral produces a result that depends upon $z$, we find that
			\begin{align*}
				M &= \rho \int_0^h dz \int_0^z r dr \int_0^{2\pi} d\theta \\
				&= \frac{\rho \pi h^3}{3} .
			\end{align*}
		To now evaluate the $z$ coordinate of the centre of mass, we evaluate 
			\begin{align*}
				\bar{z} &= \frac{1}{M} \rho \int_0^h z dz \int_0^z r dr \int_0^{2\pi} d\theta \\
				&= \frac{3}{4} h.
			\end{align*}
		We can see that both the mass and the position are dimensionally correct. \\ \\
		To find the moment of inertia, we note that the distance from the $z$ axis is just $r$, so we just evaluate the integrals as before
			\begin{align*}
				I_z &= \rho \int_0^h dz \int_0^z r^3 dr \int_0^{2\pi} d\theta \\
				&= \frac{\rho \pi h^5}{10} .
			\end{align*}
		 
	}
	\item{Find the moment of inertia of a solid ball of radius $a$ and constant density $\rho$ about the $z$ axis.\\
	The mass of a sphere is just $M = (4/3) \pi a^3 \rho$, and the distance from the $z$ axis squared is $x^2 + y^2$.  Converting to spherical coordinates gives 
		\begin{align*}
			x^2 + y^2 &= r^2 \cos^2 \phi \sin^2 \theta + r^2 \sin^2 \phi \sin^2 \theta \\
			&= r^2 \sin^2 \theta (\cos^2 \phi + \sin^2 \phi) \\
			&= r^2 \sin^2 \theta .
		\end{align*}
		Remembering that the volume element in spherical coordinates is $dV = r^2 \sin \theta dr d\theta d\phi$, we must evaluate the integral
		\begin{align*}
			I_z &= \rho \int_0^{2\pi} d\phi \int_0^{\pi} \sin^3 \theta d\theta \int_0^a r^4 dr \\
			&= \frac{8 \pi \rho a^3}{15}.
		\end{align*}
		We've skipped the steps of the integrals, but everything is pretty standard and the $\sin^3 \theta$ integral can be evaluated noting that $\sin^3 \theta = \sin \theta (1 - \cos^2 \theta)$.
	}
	%\item{Find the moment of inertia about the $z$ axis of the solid ellipsoid inside
	%	\begin{equation*}
	%		\frac{x^2}{a^2} + \frac{y^2}{b^2} + \frac{z^2}{c^2} = 1.
	%	\end{equation*}
	%}
\end{enumerate}

\end{document}